\documentclass[a4paper, 11pt]{article}
\usepackage[top=3cm, bottom=3cm, left=2cm, right=2cm]{geometry}
\usepackage[utf8]{inputenc}
\usepackage{textcomp}
\usepackage{graphicx}
\usepackage{amsmath, amssymb}
\usepackage{bm}
\usepackage[pdftex, bookmarks, colorlinks, breaklinks]{hyperref}
\usepackage{memhfixc}
\usepackage{pdfsync}
\usepackage{fancyhdr}
\usepackage{hyperref}
\usepackage{enumitem}


\title{Rapport de réunion 05 du 05/04/24}
\author{Lucien Piat, Linda Khodja, Djemilatou Ouandaogo Salamane, Maroa Alani}

\begin{document}
\maketitle
\noindent\rule{8cm}{0.4pt}
\begin{quote}
    Parties présentes : Le groupe, Le client\\
    Objectif de la réunion : Présentation de l'ébauche du cahier des charges
\end{quote}

\section{Mise à jour des Objectifs}
Compte tenu de du temps restant, le client diminue la quantité de données à traiter à 42 échantillons. 
Des lots de données seront formés pour faciliter l'analyse.\\ 

\section{Précisions apportées par le client}
\begin{itemize}[itemsep=1pt]
    \item La base de données utilisée est \href{https://fungidb.org/fungidb/app}{Fungidb}.  
    \item Bien prendre en compte que les \textit{Fusarium} SONT des phytopathogènes qui PRODUISENT des mycotoxines. 
    \item Le rendu du rapport aura lieu le 13/05 idéalement
\end{itemize}

\section{Remarques lors de la relecture}
\begin{itemize}[itemsep=1pt]
    \item Il faudra faire attention lors de l'interprétation des FastQC
    \item Le client souhaite utiliser  Cutadapt à la place de Trimmomatic
    \item Une nouvelle version de BWA, peut BWA2 être pertinente pour l'alignement sur le génome
    \item Pour l'identification des ARN il nous faudra prévoir un plan pour les reads qui ne sont pas annotés
    \item Pour la quantification des ARN, il faudra établir une stratégie de comptage
\end{itemize}

\section{Futur proche}
\begin{itemize}[itemsep=1pt]
    \item Le client relie le rapport pour le 08/04
    \item Le groupe devra envoyer une liste des métadonnées nécessaires 
    \item Recuperer les génomes de référence sur \href{https://fungidb.org/fungidb/app}{Fungidb} et bien noter la version. 
\end{itemize}

\noindent\rule{8cm}{0.4pt}
\begin{quotation}
    À l'issue de la réunion: un RDV avec le client aura lieu le 15/04
\end{quotation}

\end{document}
\documentclass[a4paper, 11pt]{article}
\usepackage[top=3cm, bottom=3cm, left=2cm, right=2cm]{geometry}
\usepackage[utf8]{inputenc}
\usepackage{textcomp}
\usepackage{graphicx}
\usepackage{amsmath, amssymb}
\usepackage{bm}
\usepackage[pdftex, bookmarks, colorlinks, breaklinks]{hyperref}
\usepackage{memhfixc}
\usepackage{pdfsync}
\usepackage{fancyhdr}

\title{Rapport de réunion 01 du 11/03/24}
\author{Lucien Piat, Linda Khodja, Djemilatou Ouandaogo Salamane, Maroa Alani}

\begin{document}
\maketitle
\noindent\rule{8cm}{0.4pt}
\begin{quote}
    Parties présentes : MBA, Le Groupe\\
    Objectif de la réunion : Mise en place du plan d'attaque pour le cahier des charges
\end{quote}

\section{Objectifs}
Afin de préciser la forme du pipeline que nous allons construire, il nous faudra nous renseigner sur les points suivants:

\subsection{Définir les Outils}
\begin{itemize}
    \item Pour faire du contrôle qualité
    \item Pour nettoyer les reads
    \item Pour aligner les reads sur les génomes
    \item Pour produire une visualisation codifiée
\end{itemize}

\subsection{Établir une Recherche Bibliographique}
\begin{itemize}
    \item À propos des organismes étudiés
    \item Sur les différents outils classiques
\end{itemize}

\section{Autres Points}
\begin{itemize}
    \item Demander au client des jeux de données au plus vite pour réaliser le test des outils
    \item Éventuellement, si l'obtention des données est trop tardive, construire un jeu test avec des données publiques
    \item Si le client le souhaite:
    \begin{itemize}
        \item Il pourrait être pertinent de construire une base de données avec les résultats
        \item Nous pourrions implémenter une option de sauvegarde des paramètres courants
    \end{itemize}
\end{itemize}

\noindent\rule{8cm}{0.4pt}
\begin{quotation}
    À l'issue de la réunion:
    \begin{itemize}
        \item La prochaine rencontre avec MBA aura lieu le 27/03
        \item Un rendez-vous avec Raluca Uricaru pourrait être pertinent.
    \end{itemize}
\end{quotation}

\end{document}
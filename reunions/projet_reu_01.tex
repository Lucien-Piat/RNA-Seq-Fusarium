\documentclass[a4paper, 11pt]{article}
\usepackage[top=3cm, bottom=3cm, left = 2cm, right = 2cm]{geometry} 
\geometry{a4paper} 
\usepackage[utf8]{inputenc}
\usepackage{textcomp}
\usepackage{graphicx} 
\usepackage{amsmath,amssymb}  
\usepackage{bm}  
\usepackage[pdftex,bookmarks,colorlinks,breaklinks]{hyperref}  
%\hypersetup{linkcolor=black,citecolor=black,filecolor=black,urlcolor=black} % black links, for printed output
\usepackage{memhfixc} 
\usepackage{pdfsync}  
\usepackage{fancyhdr}

\title{Rapport de réunion 01 du 08/03/24}
\author{Lucien Piat, Linda Khodja, Djemilatou Ouandaogo Salamane, Maroa Alani}

\begin{document}
\maketitle
\noindent\rule{8cm}{0.4pt}

\begin{quote}
    Parties présentes : Le Client, Le Groupe\\
    Objectif de la réunion : Préciser les besoins du client
\end{quote}

\section{Présentation du Sujet}

Les champignons filamenteux du genre Fusarium sont des phytopathogènes qui contaminent plusieurs types de plantes. 
Une de leurs cibles de prédilection est les céréales de blé. Ce champignon pousse partout dans les latitudes tempérées. 
Ces phytopathogènes produisent des mycotoxine très stables et se retrouvent dans les farines. Ces derniers sont aussi dangereux pour l'homme produisant des vomissements et peuvent être cancérogènes et tératogènes.\\

Les farines sont testées régulièrement pour les taux de toxines. Une trop forte dose entraîne des pertes économiques massives.\\

Ainsi, l'INRAe essaye d'étudier les mécanismes moléculaires qui conduisent à la production des toxines.
Il a été mis en évidence des communications entre les champignons du genre grâce à des miARN.\\

On essaiera donc de caractériser ces miARN produits par les champignons en coculture.

\section{Données en Entrée}
\begin{itemize}
    \item Fichiers FASTq issus de cultures. 
    \item 8 pour les cultures pures, 48 pour les croisement.
\end{itemize}

\section{Objectifs Principaux}
\begin{itemize}
    \item Avoir une liste des ARN qui sont produits par chaque espèce. Il est important d'avoir les valeurs de quantité et de qualité. 
    \item Faire un contrôle de qualité des reads.
    \item Faire un nétyoyage des séquences.
    \item Alligner les séquences sur les génomes de référence.
    \item Produire une visualisation codifiée des résultats (Fongi.db.org).
    \item Production d'un readme qui fera office de manuel.
\end{itemize}

\section{Contraintes}
\begin{itemize}
    \item Oppen source obligatoire.
    \item Faire notre travail en accès pseudo restreint.
\end{itemize}

\section{Objectifs Secondaires}
\begin{itemize}
    \item Faire des statistiques
\end{itemize}

\noindent\rule{8cm}{0.4pt}
\begin{quotation}
    A l'issue de la réunion : Un RDV avec MBA aura lieu le 11/03
\end{quotation}

\end{document}
\documentclass[a4paper, 11pt]{article}
\usepackage[top=3cm, bottom=3cm, left=2cm, right=2cm]{geometry}
\usepackage[utf8]{inputenc}
\usepackage{textcomp}
\usepackage{graphicx}
\usepackage{amsmath, amssymb}
\usepackage{bm}
\usepackage[pdftex, bookmarks, colorlinks, breaklinks]{hyperref}
\usepackage{memhfixc}
\usepackage{pdfsync}
\usepackage{fancyhdr}

\title{Rapport de réunion 04 du 04/04/24}
\author{Lucien Piat, Linda Khodja, Djemilatou Ouandaogo Salamane, Maroa Alani}

\begin{document}
\maketitle
\noindent\rule{8cm}{0.4pt}
\begin{quote}
    Parties présentes : Le groupe, MBA \\
    Objectif de la réunion : Relecture de l'ébauche du cahier des charges
\end{quote}

\section{Points à Corriger}
\subsection{Dans tous le rapport}
\begin{itemize}
    \item Changer les section en chapitres et les subsection en section pour profiter de la mise en page "report"
    \item Bien faire une bibliographie avec "cite"
    \item Parler de Bash pour la fin du projet
    \item Parler dans le document du jeu de données test
\end{itemize}

\subsection{Pour l'intro}
\begin{itemize}
    \item Résumer plus courtement le projet
    \item Mieux présenter succinctement RNAseq
    \item Ajouter les mots clés
    \item Fusionner l'intro (images, info biologique) dans le contexte
\end{itemize}

\subsection{Pour l'état de l'art}
\begin{itemize}
    \item Eviter l'effet catalogue sur les outils, il faut mettre le problème en titre et mettre les outils dans le paragraphe.  
    \item Ajouter les outils que l'on n'a pas retenu et dire pourquoi
    \item Parler des outils qui sont dans le pipeline comme miRDeep2, DESeq, edgeR
\end{itemize}

\subsection{Pour l'analyse}
\begin{itemize}
    \item Faire ressortir les deux parties distinctes dans le document, le traitement et l'analyse.
    \item Bien re préciser les besoins
\end{itemize}

\section{Autres remarques}
Ne pas hésiter à noter tout ce qu'on teste ou ce qu'on recherche pour ne pas l'oublier plus tard. Cela facilite la rédaction du rapport par la suite.

\noindent\rule{8cm}{0.4pt}
\begin{quotation}
    À l'issue de la réunion: 
    \begin{itemize}
        \item Un RDV avec MBA aura lieu le 15/04
        \item Le client relira le cahier des charges en visio le 05/04
        \item Une présentation oral du cahier aura lieu le 11/04
    \end{itemize}
\end{quotation}

\end{document}
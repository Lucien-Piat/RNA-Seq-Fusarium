\chapter*{Introduction}
\addcontentsline{toc}{chapter}{Introduction}

Inscrit dans le contexte de la recherche menée à l'INRAE (Institut National de Recherche pour l'Agriculture, l'Alimentation et l'Environnement), plus précisément dans l'unité MycSA (Mycologie et Sécurité des Aliments), ce projet vise à comprendre les mécanismes moléculaires régissant les interactions entre différentes espèces de champignons du genre Fusarium. Ces champignons filamenteux pathogènes représentent un défi majeur pour l'agriculture en raison de leur capacité à contaminer diverses cultures. Les céréales telles que le blé sont ces principales cible, ou il produit des mycotoxines nocives pour la santé humaine.\\

L'objectif de ce projet est d'analyser les séquences de petits ARNs (smARNs) produites dans chaque scénario de co-culture de différentes souches de Fusarium. D'identifier leurs caractéristiques distinctes, afin de mieux comprendre comment ces micro-organismes communiquent et comment cette communication peut influencer la production de toxines.\\

Pour ce faire, nous mettrons en place un pipeline bioinformatique intégrant différentes étapes d'analyse, du prétraitement des données à la visualisation des résultats.\\

Ce rapport détaille les objectifs, la méthodologie, et les résultats attendus de notre projet. Nous mettons en contexte l'importance biologique des espèces de Fusarium dans les domaines agricoles et alimentaires, ainsi que la communication au sein de la méta-communauté de Fusarium et le rôle des smARNs, notamment les miARNs, dans cette dynamique. Nous décrivons également le processus analytique, incluant le prétraitement des données, l'alignement, l'identification, la quantification, et l'analyse de l'expression différentielle des miARNs.\\

Notre étude vise à contribuer à la connaissance des Fusarium et à la prévention des mycotoxines dans les céréales, pour garantir la productivité agricole et la santé humaine.Pour relever ces défis, nous avons structuré notre enquête comme suit, en commençant par une analyse approfondie du contexte biologique et de l'état actuel de la recherche.\\

\textbf{Mots-clés} : Fusarium, smARN, miARN RNA-Seq, pipeline bioinformatique, mycotoxines, co-culture, meta-Fusarium, communication.\\